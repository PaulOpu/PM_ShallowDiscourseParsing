\documentclass[10pt,notitlepage]{scrartcl}
\usepackage[a4paper,margin=1.5in]{geometry}
\usepackage[utf8]{inputenc}
\usepackage{amsmath}
\usepackage{amsfonts}
\usepackage{amssymb}
\usepackage{bm}
\usepackage{moreverb}
\usepackage{graphicx}
\usepackage{setspace}

\usepackage{etoolbox}
\patchcmd{\thebibliography}{\chapter*}{\section*}{}{}
\AtBeginEnvironment{quote}{\singlespacing\small}
\DeclareMathOperator{\sgn}{sgn}


\author{Paul Opuchlich, 794745\\Malte Klingenberg, 794394}
\title{Shallow Discourse Parser Comparison and Combination}
\date{}
\begin{document}
\pagenumbering{gobble}
\maketitle

\section*{Introduction}
lot of development in discourse parsing (short overview?)

parsers have gotten better, but are still far from perfect

many parsers available, maybe some better at some things than others

goal: compare parsers, find a way to combine them to give better results
\section*{Dataset}
CoNLL challenge \cite{xue2016conll}

incomplete data (sense-only), generate dummy data (see below)

used parsers (differences, some only guess a limited number of senses)
\section*{Approach}
Preprocessing and Randomising

Mapping the Relations (using CoNLL methods)

Calculate Parameters for each Parser + Relation, some evaluation

Create combined model by different voting approaches (best-wins, weighted...)
\section*{Evaluation}
general overview

agreement between three parsers

compare performances of all parsers

relationship between architecture and results?

combination (different approaches)

\section*{Conclusion}
parsers are(n't) good at different things

combination (does not) improve results

\pagebreak
\bibliography{projectreport}
\bibliographystyle{apalike}

\end{document}