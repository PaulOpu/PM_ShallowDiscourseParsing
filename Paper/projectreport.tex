\documentclass[10pt,notitlepage]{scrartcl}
\usepackage[a4paper,margin=1.5in]{geometry}
\usepackage[utf8]{inputenc}
\usepackage{amsmath}
\usepackage{amsfonts}
\usepackage{amssymb}
\usepackage{bm}
\usepackage{moreverb}
\usepackage{graphicx}
\usepackage{setspace}

\usepackage{etoolbox}
\patchcmd{\thebibliography}{\chapter*}{\section*}{}{}
\AtBeginEnvironment{quote}{\singlespacing\small}
\DeclareMathOperator{\sgn}{sgn}


\author{Paul Opuchlich, 794745\\Malte Klingenberg, 794394}
\title{Shallow Discourse Parser Comparison and Combination}
\date{}
\begin{document}
\pagenumbering{gobble}
\maketitle

%%%%%%%%%%%%%%%%%%%%%%%%%%%%%%%%%%%%%%%%%%%%%%%%%%%%%%%%%%%%%%%%
%%%%% INTRODUCTION %%%%%%%%%%%%%%%%%%%%%%%%%%%%%%%%%%%%%%%%%%%%%
%%%%%%%%%%%%%%%%%%%%%%%%%%%%%%%%%%%%%%%%%%%%%%%%%%%%%%%%%%%%%%%%
\section*{Introduction}
The field of text discourse theory has been around for several decades, with first theoretical approaches such as Rhetorical Structure Theory (RST) developed in the 1980s \cite{mann1988rhetorical}. Advancements in machine learning have fueled the development of many different kinds of discourse parsers, but while significant improvements have been made in recent years, the existing systems are still far from perfect. Explicit discourse relations (those marked by a connective such as ''and'', ''because'', ''after'' etc.) can be identified by modern systems with high accuracy, but implicit and entity relations pose a far greater challenge. Because the argument spans of discourse relations can vary wildly (from single clauses to several paragraphs), identifying them is another difficult task.

Because the many available parsers are based on different machine learning algorithms, for example support vector machines (SVMs) and different types of neural networks such as convolutional (CNN) and recurrent (RNN), there might be systematical differences in their performances on different tasks. It is not unreasonable to assume that, for example, one machine learning algorithm might perform better at classifying implicit senses, while another algorithm might be best suited to identifying argument spans.

Our goal in this project was to compare the performances of several different parsers and check whether any systematic patters can be found. We also tried to find a way to combine several parsers in a way that gives better results than the individual parsers to see whether a ''crowd intelligence'' approach of combining different systems can be a good way to improve performance.

%%%%%%%%%%%%%%%%%%%%%%%%%%%%%%%%%%%%%%%%%%%%%%%%%%%%%%%%%%%%%%%%
%%%%% DATASET %%%%%%%%%%%%%%%%%%%%%%%%%%%%%%%%%%%%%%%%%%%%%%%%%%
%%%%%%%%%%%%%%%%%%%%%%%%%%%%%%%%%%%%%%%%%%%%%%%%%%%%%%%%%%%%%%%%
\section*{Dataset and Parsers}
For our project, we used data from the 2016 CoNLL shared task \cite{xue2016conll}. We were graciously provided with the output of all submitted parsers for the sense-only challenge, but could not obtain full data including the argument spans. Because only few parsers were publicly available and we were only able to run two \textbf{[CHECK]} of them, we limited our analysis to sense identification.

The methods we used for analysing the sense identification performance can also be used for the argument spans. Below, we present a way to create dummy data for testing purposes. If full data should become available, the analysis can easily be extended to include the argument spans.

The parsers submitted to the CoNLL 2016 shared task differ in their scope. Some limited themselves to a certain subtask, such as identifying explicit and implicit senses. For our purposes, we included all parsers that completed the sense-only task. A list of them can be found in the evaluation chapter. We will compare the performance of our combination approaches against the highest-scoring parser in the overall task, the oslopots parser \cite{oepen2016opt}.

For our combination approaches, we used the output of the parsers on the CoNLL 2016 shared task ''test'' set as training data and evaluated them and the other parsers on the ''blind test'' set.

%%%%%%%%%%%%%%%%%%%%%%%%%%%%%%%%%%%%%%%%%%%%%%%%%%%%%%%%%%%%%%%%
%%%%% APPROACH %%%%%%%%%%%%%%%%%%%%%%%%%%%%%%%%%%%%%%%%%%%%%%%%%
%%%%%%%%%%%%%%%%%%%%%%%%%%%%%%%%%%%%%%%%%%%%%%%%%%%%%%%%%%%%%%%%
\section*{Our Approach}
In a first short preprocessing step, we standardised the parser outputs, since some of them differed in minor technical aspects, such as treatment of the ''Connective'' argument for relations without a connective (implicit relations and EntRels). To generate additional data for testing, we developed a simple randomiser which takes the gold standard relation file and applies some random modifications to it. This may be in the form of changing the identified sense or randomly extending or reducing one or both of the argument spans.

In the next step, the parser outputs are scored according to the CoNLL shared task partial scoring procedure. Since the parsers may miss some relations present in the gold standard and falsely identify relations not found in the gold standard, the gold and parsed relations are aligned to be able to score them accurately. Further details on the aligning step can be found in the CoNLL shared task documentation.

The precision, recall and F1 scores are then calculated for each parser for all relations using the CoNLL shared task methods. We first used these scores to evaluate a possible relationship between parser architecture and performance on different relation types as detailed above.

We then used the F1 scores to implement several methods of combining the parser outputs, which were:
\begin{itemize}
\item \textbf{best wins:} For each pair of argument spans, check the relation sense identified by each parser and its F1 score for that sense. Choose the sense with the highest F1 score.
\item \textbf{agreement:} For each pair of argument spans, check the relation sense identified by each parser. Choose the sense which most parsers chose (simple majority voting).
\item \textbf{probability maximisation:} For each pair of argument spans, check the relation sense identified by each parser and its F1 score for that relation. For each sense, add the F1 scores of all parsers which chose that sense. Chose the sense with the highest aggregated score.
\item \textbf{best three agreement:} Same as agreement, but only take into account the three identified senses with the highest F1 score.
\end{itemize}

%%%%%%%%%%%%%%%%%%%%%%%%%%%%%%%%%%%%%%%%%%%%%%%%%%%%%%%%%%%%%%%%
%%%%% EVALUATION %%%%%%%%%%%%%%%%%%%%%%%%%%%%%%%%%%%%%%%%%%%%%%%
%%%%%%%%%%%%%%%%%%%%%%%%%%%%%%%%%%%%%%%%%%%%%%%%%%%%%%%%%%%%%%%%
\section*{Evaluation}
As a first step, we looked at the inter-parser agreement to get a grasp on the individual and group performance of the parsers. Table \ref{tab:comp1} shows the inter-parser sense agreement between the three highest scoring parsers ''oslopots'', ''ecnucs'' and ''steven''.

It is rare that all parsers predict the correct sense, with this happening for a third of all occurences for a contrast relation, but never for restatements. It is far more common that one or two of the parsers get the correct result, with the others predicting a wrong sense. For all relations, this happens in at least half of the cases, with percentages usually being higher than that, for example 200 out of 217 for entity relations and 327 out of 391 for conjunctions.

There are, however, clear differences in difficulty of recognition between the senses. For concessions, results, and restatements, it is more common that all parsers give a wrong result than even a single parser giving a correct one. Apparently, these senses have structures that make them harder to identify for the algorithms. This is consistent with the results we found later, where these senses usually had a lower-than-average or even the lowest F1 score among all senses.

An interesting thing to note is that it sometimes occurs than all three parsers agree on the same wrong sense. This is especially common for condition (7 out of 12) and succession (14 out of 24) relations. \textbf{WHAT DO THEY AGREE ON INSTEAD?}

%%%%%%%%%%%%%%%%%%%%%%%%%%%%%%%%%%%%%%%%%%%%%%%%%%%%%%%%%%%%%%%%
\begin{table}[htbp]
\centering
\resizebox{\textwidth}{!}{%
\begin{tabular}{l|r|r|r|r|r}
sense & total & all correct & at least one correct & all wrong & same wrong \\ \hline

Comp.Concession 	 & 33  & 3   & 14  & 19  & 4  \\
Comp.Contrast 		 & 399 & 131 & 296 & 103 & 20 \\
Cont.Cause.Reason 	 & 195 & 27  & 120 & 75  & 10 \\
Cont.Cause.Result 	 & 135 & 15  & 60  & 75  & 5  \\
Cont.Condition 		 & 63  & 30  & 51  & 12  & 7  \\
EntRel 				 & 217 & 58  & 200 & 17  & 2  \\
Exp.Alternative 	 & 5   & 0   & 5   & 0   & 0  \\
Exp.Conjunction 	 & 391 & 103 & 327 & 64  & 13 \\
Exp.Instantiation    & 91  & 14  & 56  & 35  & 4  \\
Exp.Restatement 	 & 198 & 0   & 84  & 114 & 9  \\
Temp.Asyn.Precedence & 45  & 19  & 38  & 7   & 0  \\
Temp.Asyn.Succession & 69  & 15  & 45  & 24  & 14 \\
Temp.Synchrony 		 & 76  & 16  & 61  & 15  & 3  \\
\end{tabular}}
\caption{Sense agreement (implicit and explicit) between oslopots, ecnucs and steven}
\label{tab:comp1}
\end{table}
%%%%%%%%%%%%%%%%%%%%%%%%%%%%%%%%%%%%%%%%%%%%%%%%%%%%%%%%%%%%%%%%

\textit{individual evaluation and pairwise agreement}

\textit{compare performances of all parsers}

\textit{relationship between architecture and results?}

The results of our combination approaches are shown in tables \ref{tab:ours1} and \ref{tab:ours2}. The ''best wins'' approach gave the worst results, with the overall F1 score being 11\% lower than the oslopots results. This might be because selecting only one parser can skip over the good parsers and select one that gave good results in training, but performs poorly on the test data. The other results are all close to the oslopots score. Probability maximisation and best three agreement both give slightly worse results than the oslopots parser, probably due to the same problems as mentioned above.

The agreement method however, where a simple majority voting between the parsers is utilised, improves the overeall F1 score by about 1\%. While the F1 scores for individual relation senses might decrease slightly (not more than 1\%), for some senses it increases by several percent, for example for Temporal.Asynchronous.Precedence (0.8261 $\rightarrow$ 0.8571). \textbf{[EXPAND?]}

While the improvements made by combining the parsers are not huge, they are significant and can be used if further slight improvement of the results is desired.

%%%%%%%%%%%%%%%%%%%%%%%%%%%%%%%%%%%%%%%%%%%%%%%%%%%%%%%%%%%%%%%%
\begin{table}[htbp]
\centering
\resizebox{\textwidth}{!}{%
\begin{tabular}{l|l|l|l|l|l|l|l|l|l|l|l|}
 & steven & oslopots & ecnucs & tao0920 & goethe & nguyenlab & clac & PurdueNLP & gw0 & ykido & gtnlp\\ \hline
sense & - & SVM & linear, CNN & SVM, CNN & SVM, FF-NN & random forest & CRF, CNN & SVM, FF-NN & RNN & SVM & - \\ \hline
Comp.Concession 	 & 0.0746\\
Comp.Contrast 		 & 0.2560\\
Cont.Cause.Reason 	 & 0.1429\\
Cont.Cause.Result 	 & 0.1557\\
Cont.Condition 		 & 0.2937\\
EntRel 				 & 0.1943\\
Exp.Alternative 	 & 0.2500\\
Exp.Conjunction 	 & 0.2176\\
Exp.Instantiation    & 0.1339\\
Exp.Restatement 	 & 0.0148\\
Temp.Asyn.Precedence & 0.2524\\
Temp.Asyn.Succession & 0.2252\\
Temp.Synchrony 		 & 0.1701\\ \hline
\end{tabular}}
\caption{text}
\label{tab:comp4}
\end{table}
%%%%%%%%%%%%%%%%%%%%%%%%%%%%%%%%%%%%%%%%%%%%%%%%%%%%%%%%%%%%%%%%
\begin{table}[htbp]
\centering
\resizebox{\textwidth}{!}{%
\begin{tabular}{l|c|c|c||c|c|c||c|c|c|}
\multicolumn{1}{ }{ } & \multicolumn{3}{|c||}{oslopots} & \multicolumn{3}{|c||}{best wins} & \multicolumn{3}{c|}{agreement} \\
sense & P & R & F & P & R & F & P & R & F \\ \hline

*Micro-Average 		 & 0.5485 & 0.5476 & 0.5480 & 0.4334 & 0.4334 & 0.4334 & 0.5559 & 0.5550 & 0.5555\\
Comp.Concession 	 & 1.0000 & 0.0660 & 0.1239 & 1.0000 & 0.0000 & 0.0000 & 1.0000 & 0.0660 & 0.1239\\
Comp.Contrast 		 & 0.2160 & 0.4909 & 0.3000 & 0.1360 & 0.5636 & 0.2191 & 0.2500 & 0.4909 & 0.3313\\
Cont.Cause.Reason 	 & 0.4267 & 0.4384 & 0.4324 & 0.4688 & 0.2027 & 0.2830 & 0.4051 & 0.4384 & 0.4211\\
Cont.Cause.Result 	 & 0.6000 & 0.3000 & 0.4000 & 0.5000 & 0.0204 & 0.0392 & 0.5926 & 0.3200 & 0.4156\\
Cont.Condition 		 & 0.8667 & 1.0000 & 0.9286 & 0.7879 & 1.0000 & 0.8814 & 0.9286 & 1.0000 & 0.9630\\
EntRel 				 & 0.4306 & 0.7600 & 0.5497 & 0.4151 & 0.2200 & 0.2876 & 0.4262 & 0.7800 & 0.5512\\
Exp.Alternative 	 & 1.0000 & 0.3333 & 0.5000 & 1.0000 & 0.3333 & 0.5000 & 1.0000 & 0.3333 & 0.5000\\
Exp.Conjunction 	 & 0.6704 & 0.7368 & 0.7021 & 0.4595 & 0.8369 & 0.5932 & 0.6839 & 0.7368 & 0.7094\\
Exp.Instantiation    & 0.6000 & 0.1364 & 0.2222 & 0.4762 & 0.2273 & 0.3077 & 0.5455 & 0.1364 & 0.2182\\
Exp.Restatement 	 & 0.4923 & 0.2133 & 0.2977 & 0.6667 & 0.0132 & 0.0260 & 0.4789 & 0.2267 & 0.3077\\
Temp.Asyn.Precedence & 0.9048 & 0.7600 & 0.8261 & 0.5063 & 0.8000 & 0.6202 & 0.9512 & 0.7800 & 0.8571\\
Temp.Asyn.Succession & 0.9600 & 0.7500 & 0.8421 & 0.9216 & 0.7344 & 0.8174 & 0.9600 & 0.7500 & 0.8421\\
Temp.Synchrony 		 & 0.5538 & 0.6792 & 0.6102 & 0.5439 & 0.6200 & 0.5794 & 0.5606 & 0.6981 & 0.6218\\ \hline
Overall 			 & 0.5485 & 0.5476 & 0.5480 & 0.4334 & 0.4334 & 0.4334 & 0.5559 & 0.5550 & 0.5555\\ \hline
\end{tabular}}
\caption{Performance comparison between the oslopots parser and our combination approaches. Senses are abbreviated for space reasons.}
\label{tab:ours1}
\end{table}

\begin{table}[htbp]
\centering
\resizebox{\textwidth}{!}{%
\begin{tabular}{l|c|c|c||c|c|c||c|c|c|}
\multicolumn{1}{ }{ } & \multicolumn{3}{|c||}{oslopots} & \multicolumn{3}{|c||}{prob. maximisation} & \multicolumn{3}{c|}{best three agreement} \\
sense & P & R & F & P & R & F & P & R & F \\ \hline

*Micro-Average 		 & 0.5485 & 0.5476 & 0.5480 & 0.5360 & 0.5352 & 0.5356 & 0.5423 & 0.5409 & 0.5416 \\
Comp.Concession 	 & 1.0000 & 0.0660 & 0.1239 & 1.0000 & 0.0660 & 0.1239 & 0.8571 & 0.0566 & 0.1062 \\
Comp.Contrast 		 & 0.2160 & 0.4909 & 0.3000 & 0.2347 & 0.4182 & 0.3007 & 0.2090 & 0.5091 & 0.2963 \\
Cont.Cause.Reason 	 & 0.4267 & 0.4384 & 0.4324 & 0.4706 & 0.4384 & 0.4539 & 0.4756 & 0.5342 & 0.5032 \\
Cont.Cause.Result 	 & 0.6000 & 0.3000 & 0.4000 & 0.5455 & 0.2449 & 0.3380 & 0.4571 & 0.3200 & 0.3765 \\
Cont.Condition 		 & 0.8667 & 1.0000 & 0.9286 & 0.9286 & 1.0000 & 0.9630 & 0.8966 & 1.0000 & 0.9455 \\
EntRel 				 & 0.4306 & 0.7600 & 0.5497 & 0.3824 & 0.7150 & 0.4983 & 0.4238 & 0.6400 & 0.5100 \\
Exp.Alternative 	 & 1.0000 & 0.3333 & 0.5000 & 1.0000 & 0.3333 & 0.5000 & 1.0000 & 0.3333 & 0.5000 \\
Exp.Conjunction 	 & 0.6704 & 0.7368 & 0.7021 & 0.6512 & 0.7377 & 0.6918 & 0.6817 & 0.7469 & 0.7128 \\
Exp.Instantiation    & 0.6000 & 0.1364 & 0.2222 & 0.3750 & 0.0682 & 0.1154 & 0.6111 & 0.2500 & 0.3548 \\
Exp.Restatement 	 & 0.4923 & 0.2133 & 0.2977 & 0.4512 & 0.2450 & 0.3176 & 0.4267 & 0.2133 & 0.2844 \\
Temp.Asyn.Precedence & 0.9048 & 0.7600 & 0.8261 & 0.9737 & 0.7400 & 0.8409 & 0.6909 & 0.7600 & 0.7238 \\
Temp.Asyn.Succession & 0.9600 & 0.7500 & 0.8421 & 0.9592 & 0.7344 & 0.8319 & 0.9066 & 0.7500 & 0.8421 \\
Temp.Synchrony 		 & 0.5538 & 0.6792 & 0.6102 & 0.5902 & 0.6923 & 0.6372 & 0.5932 & 0.6731 & 0.6306 \\ \hline
Overall 			 & 0.5485 & 0.5476 & 0.5480 & 0.5360 & 0.5352 & 0.5356 & 0.5423 & 0.5409 & 0.5416\\ \hline
\end{tabular}}
\caption{Performance comparison between the oslopots parser and our combination approaches. Senses are abbreviated for space reasons.}
\label{tab:ours2}
\end{table}
%%%%%%%%%%%%%%%%%%%%%%%%%%%%%%%%%%%%%%%%%%%%%%%%%%%%%%%%%%%%%%%%
%%%%% CONCLUSION %%%%%%%%%%%%%%%%%%%%%%%%%%%%%%%%%%%%%%%%%%%%%%%
%%%%%%%%%%%%%%%%%%%%%%%%%%%%%%%%%%%%%%%%%%%%%%%%%%%%%%%%%%%%%%%%
\section*{Conclusion}
parsers are(n't) good at different things

combination (does not) improve results

\pagebreak
\bibliography{projectreport}
\bibliographystyle{apalike}

\end{document}